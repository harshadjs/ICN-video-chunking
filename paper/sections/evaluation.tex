\section{Evaluation} \label{sec:eval}

\subsection{Metrics Evaluated} \label{sec:metrics}

Join Time: This metric was used in the QoE paper as a measure of user
experience. It refers to the time it takes for the video to start playing after
it has been requested. Because we also used a video-based workload, we believed
this was still a very relevant metric.

Buffering Ratio: This metric was also used in the QoE paper. This is the ratio
between the  time that the user spends waiting for the video to buffer and the
total time to finish watching the video. We felt this would also be an
interesting metric, because we expected it to vary inversely with chunk size, as
it would take longer for increments of the video to be buffered. However, it is
possible that our buffer size is large enough that we prefetch enough chunks
that this is not the case.

Average Response Time: Although this is not  really a quality of experience
metric, we felt that measuring the average response time of interests would give
some insight into the effects of caching. If the pending interest table is
large, we would expect the average response time to increase. By the definition
of our network, with no delays whatsoever, the upper bound on the ART is 40
milliseconds, because we defined each of the links to have 10 millisecond
latency.

Pending Interest Table Size: This structure is import to us because we expected
the number of pending interests to vary inversely with the chunk size. With
smaller chunks, chunk request are made more frequently because the buffer
empties in smaller increments, so more requests should pile up on the router
node.

Server Load Reduction: This is fraction of requests that was served by the cache
rather than the server. This is a standard way to measure the benefits to the
server due to caching.

\subsection{Chunk Size} \label{sec:chunksize}

We ran experiments for the following chunk sizes (in bytes): 128, 256, 512,
1024, 4096, 8192.

