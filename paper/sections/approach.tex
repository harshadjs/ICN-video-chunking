\section{Approach} \label{sec:approach}
We discussed two different possibilities for running our experiments. The first
option we considered was using real hardware to run the client, router, and
server nodes for our experiments. The advantage of this approach is that it
would be as close to a real, albeit very small, network in the real world. The
obvious downside to this approach is the difficulty in setting up experiments,
especially in implementing caching in commercial routers if we wanted to go all
the way with this approach. There are testbeds like emulab which we had
previous experience using that allow experiments to easily define physical nodes
and links in a network, but these do not provide any customization capabilities
for the routers involved in the links. We quickly decided not to pursue this
route.

The second approach, which was ultimately better suited to our purposes, is a
simulation and trace-driven approach, similar to [2]. Using simulation software
allowed us to run many different configurations of experiments on individual
machines, and to predefine workloads that we could keep constant across runs.
Because we did not have to worry about acquiring and  transferring real files in
our experiments, we could instead just simulate their effects on the network.
This approach was easier than the former, while still being able to produce
valid results.

\subsection{ndnSIM} \label{sec:ndnSIM}
ndnSIM is a NS-3 based simulator for Named Data Networking (NDN), a networking
model very similar to Content Centric Networking (CCN). It allows users to set
up custom topologies, and define the behavior of individual nodes in the network
using custom applications. In terms of our intended use of simulating the
transfer of chunks between servers and clients in a network, ndnSIM is perfect
because it already abstracts these content chunks into data and interests.
Another big draw of ndnSIM is that it already has an implementation of a cache
in the form of a Content Store structure. These Content Store objects are more
general than the router caches we want to test, because they can be placed on
any nodes in the network (not just routers). They are minimally configurable,
allowing users to set the eviction policy and the max size in terms of packets. 
