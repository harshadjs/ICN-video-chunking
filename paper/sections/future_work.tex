\section{Future Work} \label{sec:futurework}

In order to obtain some early results and verify that our experimentation
platform worked, we had to make some simplifications to our experiments. If we
had more time, we would have liked to explore the following improvements to our
model.

\subsection{Topologies} \label{sec:topologies}

We used a very rudimentary topology in the experiments evaluated in this
project. We would have ideally liked to run our experiments on a topology more
representative of a real network. Rocketfuel provides some PoP-level ISP maps
which we could try to emulate in future experiments \cite{3}. Of course, with a larger
topology, is it more difficult to collect data and then draw conclusion based on
factors like connectedness of the router and how far it is from the edge.

\subsection{User behavior} \label{sec:userbehavior}

We would also like to try make our model account for real-world user behavior
better. For example, in our experiments, we had a very simple view of user
viewing behavior. We assumed the user had a uniform probability of stopping the
video at any point from the start to the end. While this accounts for the fact
that not all videos are watched in entirely, in reality, people are more likely
to watch the front part of the video than the end. Also, people do not
necessarily start watching the video from the beginning and may skip around.
Although these factors might not drastically affect our results, if we accounted
for them, we could say our experiments more closely model the real world.

\subsection{non-simulated Video Traces} \label{sec:traces}

For the sake of our experiments, we generated an artificial workload based on a
zipf distribution of popularity and an arbitrary range of lengths and bitrate.
In future experiments, we would like to parse and interpret some real traces of
video watching in order to compute the effects of chunk size on them. 

