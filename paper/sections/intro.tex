\section{Introduction} \label{sec:Intro}

The idea of introducing caching to network routers is not a novel one. As the
type of network traffic shifted from mostly host-to-host connections to content
based traffic, it became clear that caching could provide savings as popular
content was requested frequently by many different clients. Some recent work by
Sun et al focuses on streaming video as a major source of traffic in modern
networks [2], and as the size of these videos and available user bandwidth both
increase today, we will see more and more redundant data make it’s way across
the internet. By introducing network caches, it has been shown that a
significant reduction in network traffic is achievable. However, while these
publications take into account many different aspects of router caches such as
cache placement and size, they sidestep a potentially important parameter: the
size of the individual chunks themselves. 

In this project, we explore the effects of chunk size on a number of previously
established metrics for both user quality of experience and server performance.
More specifically, we measure DUBBLE CHECK {join-time, buffering ratio, average
latency of requests, and server load reduction}. Using the ndnSIM module for the
NS-3 network simulator, we designed a client and server application to quantify
these metrics. We run our experiments on a small, straightforward topology which
while not accurately representative of the real world, demonstrates the type of
results we would expect in practice. 

